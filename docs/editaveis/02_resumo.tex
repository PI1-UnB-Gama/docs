\begin{resumo}
%\textcolor{red}{O resumo é um item obrigatório. Essa parte do relatório será uma visão rápida e clara do projeto desenvolvido. O leitor terá informações como: descrição breve do projeto, principais requisitos, tecnologias necessárias e outras informações relevantes para apresentação. O resumo terá no máximo meia (1/2) página.}
O seguidor de linha é uma categoria de robôs autônomos geralmente semelhantes a carros de corrida cujo objetivo é realizar um percurso no menor tempo possível.
O carrinho seguidor de linha descrito nesse projeto vai um pouco além, além de ser capaz de realizar três (3) percursos aleatórios no menor tempo possível, o carrinho precisa levar em si um ovo, que deve permanecer intacto ao final do percurso. 
Para isso, o projeto irá utilizar uma combinação de sensores, motores, algoritmos computacionais (incluindo Inteligência Artificial) e um arduíno de forma que o robô não saia da linha a ser disposta no chão e possa também calcular sua velocidade e tempo de conclusão do percurso.
Esperamos com esse trabalho não só impulsionar a capacidade de trabalho em equipe de estudantes de diversas engenharias como também apresentar aos mesmos diversas novas dimensões de aplicação desse mesmo produto nas indústrias, aumentar o entendimento da influência de outras engenharias na nossa própria e melhorar a capacidade de resolução de problemas e imprevistos.


%\textcolor{red}{Ao longo deste texto, as descrições em \textbf{cor vermelha} são meras instruções, não devendo aparecer na versão final do texto. \textbf{Utilize o comentário \% ao invés de apagar estas descrições, para não perder as orientações apresentadas.}}

\vspace{\onelineskip}
\noindent
\textbf{Palavras-chaves}: \textcolor{red}{ \imprimirpalavrachaveum, \imprimirpalavrachavedois}
\end{resumo}

% \begin{resumo}
%  O resumo deve ressaltar o objetivo, o método, os resultados e as conclusões 
%  do documento. A ordem e a extensão
%  destes itens dependem do tipo de resumo (informativo ou indicativo) e do
%  tratamento que cada item recebe no documento original. O resumo deve ser
%  precedido da referência do documento, com exceção do resumo inserido no
%  próprio documento. (\ldots) As palavras-chave devem figurar logo abaixo do
%  resumo, antecedidas da expressão Palavras-chave:, separadas entre si por
%  ponto e finalizadas também por ponto. O texto pode conter no mínimo 150 e 
%  no máximo 500 palavras, é aconselhável que sejam utilizadas 200 palavras. 
%  E não se separa o texto do resumo em parágrafos.

%  \vspace{\onelineskip}
    
%  \noindent
%  \textbf{Palavras-chave}: latex. abntex. editoração de texto.
% \end{resumo}
