\begin{apendicesenv}
% \partapendices

\chapter{Primeiro Apêndice}

\textcolor{red}{Apêndices e anexos são materiais complementares ao texto que só devem ser incluídos quando forem imprescindíveis à compreensão deste:}

%%% https://www.codecogs.com/latex/eqneditor.php?lang=pt-br
% \begin{equation}
%     \alpha = \sum_{i=0}^{\infty}{a_i x^i}
% \end{equation}

% \begin{equation}
%     \beta = \sum_{i=0}^{\infty}{a_i x^{2\pi i}\frac{\partial^2 }{\partial x^2}}f(xi)
% \end{equation}

\textcolor{red}{
\begin{itemize}
    \item Apêndices são textos elaborados pelo autor a fim de complementar sua argumentação.
    \item Anexos são os documentos não elaborados pelo autor, que servem de fundamentação, comprovação ou ilustração, como mapas, leis, estatutos etc.
\end{itemize}}

\textcolor{red}{\textbf{Texto do primeiro apêndice.}}

\chapter{Segundo Apêndice}
\textcolor{red}{Apêndices e anexos são materiais complementares ao texto que só devem ser incluídos quando forem imprescindíveis à compreensão deste:}

\textcolor{red}{
\begin{itemize}
    \item Apêndices são textos elaborados pelo autor a fim de complementar sua argumentação.
    \item Anexos são os documentos não elaborados pelo autor, que servem de fundamentação, comprovação ou ilustração, como mapas, leis, estatutos etc.
\end{itemize}}

\textcolor{red}{\textbf{Texto do segundo apêndice.}}

\end{apendicesenv}