\chapter{Projeto Conceitual do Produto}

\section{Características gerais}

\textcolor{red}{Descrever produto de forma geral. Apresentar itens teóricos sobre o projeto a serem aprofundados ou detalhados oportunamente.}

\textcolor{red}{Apresentar e explicar a Estrutura Analítica do Projeto (EAP), tomando cuidado com a legibilidade da imagem. Se necessário, coloque a EAP dentro do comando ``\textsf{\textbackslash begin\{landscape\} \textbackslash end\{landscape\}}'' para que ela seja apresentada em uma página deitada.}

\section{Estrutura}

\textcolor{red}{Apresentar desenho em CAD, indicar dimensões por lado/aresta/cota, indicar materiais utilizados e explicar o desenho e as decisões de projeto.}

\section{Descrição de \textit{hardware}}

\textcolor{red}{A descrição de hardware no documento deve permitir que outras pessoas repliquem o produto proposto neste documento. Para isso, indiquem:}

\textcolor{red}{\begin{itemize}
    \item O diagrama de blocos \cite{blockdiagram}, que oferece uma visão geral do hardware, com as principais partes do sistema e as ligações entre elas.
    \item A lista de materiais \cite{bom}, que facilita a montagem do sistema, indicando tudo que deve ser adquirido. Aproveitem a lista de materiais para apresentarem o orçamento dos mesmos.
    \item O esquemático \cite{esquematico}, que mostra as conexões elétricas entre os componentes. Deve-se indicar os componentes por nomes e símbolos, e as conexões entre eles, incluindo a pinagem dos componentes. Se o esquemático ficar muito grande, pode-se separa-lo em vários esquemáticos. A ligação em \textit{protoboard} não é considerada um esquemático adequado.
\end{itemize}}

\textcolor{red}{Além de mostrar tudo isso, é necessário organizar a descrição de \textit{hardware} no texto, e explicar as escolhas feitas. Comece com o diagrama de blocos, que é mais genérico, e depois indiquem os detalhes com a lista de materiais e os esquemáticos.}

\section{Análise de consumo energético}

\textcolor{red}{Com relação ao consumo energético do produto desenvolvido, será necessário apresentar cálculos de consumo dos diversos componentes utilizados e explicar as decisões de projeto para atender a estas demandas.}

\section{Descrição de \textit{software}}

\textcolor{red}{Com relação ao \textit{software}, será necessário apresentar os seguintes itens: % pacotes de componentes de \textit{software}, suas funções e características, e explicar as decisões de projeto:
\begin{enumerate}
    \item Um diagrama do processo de negócio do problema que a máquina se propõe a resolver (BPNM) – não é UML, mas é fundamental para entender como o sistema se comporta como um todo, incluindo o usuário;
    \item Lista de casos de uso (backlog do sistema). Backlog funcional;
    \item Lista de requisitos não-funcionais a serem satisfeitos pelo sistema;
    \item Diagrama de casos de uso: mostrando os requisitos funcionais, seus atores e como eles interagem entre si;
    \item Diagrama de Classes: apresentando quais dados são manipulados pelo sistema (internamente e externamente – ex.: resultados de experimentos);
    \item Diagrama de arquitetura, identificando todos os componentes da máquina e suas iterações com o software;
    \item Diagrama de estados da máquina (sistema);
    \item Descrição dos testes dos componentes da máquina e dos testes funcionais que deveriam ser feitos para avaliar o funcionamento da máquina e identificar defeitos. São importantes os testes unitários (componentes) e de integração (conjunto de componentes) e o roteiro de testes.
\end{enumerate}
}
%\textcolor{red}{Com relação ao \textit{software}, será necessário apresentar pacotes de componentes de \textit{software}, suas funções e características, e explicar as decisões de projeto.}

