\chapter[Introdução]{Introdução}

Os projetos de robôs seguidores de linha são uma ótima forma de introduzir tanto grupos de estudantes ou entusiastas interessados em projetos DIY (\textit{Do It Yourself} ou Faça Você Mesmo) ao mundo da engenharia. Isto pois este é um projeto relativamente simples e que consegue combinar diversos conceitos de engenharia, automação e programação. Além desses conceitos, quando feito em grupo, o projeto tem a capacidade de ajudar no desenvolvimento pessoal dos integrantes em áreas como trabalho em equipe, resolução de problemas e também instiga o espírito competitivo dos integrantes. Falando sobre espírito esportivo, atualmente existem diversas competições estudantis tanto nacionais quanto internacionais onde são realizadas provas que testam a capacidade autônoma do robô em paralelo com a capacidade do mesmo realizar a prova no menor tempo possível. Mas essa não é a única aplicação desse tipo de projeto, ele também possui um grande potencial de aplicações no mundo real, onde os mesmos conceitos podem ser encontrados em robôs autônomos que são utilizados em linhas de produção de grandes indústrias.

Uma vez que os robôs têm, frequentemente, um propósito de competição, a diretriz que o projeto deve seguir varia da competição ou propósito final. No Brasil, não existem quaisquer proibições sobre pequenos robôs seguidores de linha, as possíveis legislações que podem abranger esse robô variam de acordo com seu propósito, como se ele irá ou não ser comercializado (Lei de Segurança de Produtos - Lei nº 8.078/1990), se ele utiliza ou não formas sem fio para sua partida (conformidades da ANATEL) e, se possuindo uma câmera, ela faz registro e armazenamento das gravações (Lei Geral de Proteção de Dados - Lei nº 13.709/2018). Por fim, é responsabilidade dos construtores do robô não infringir leis como a Lei de Propriedade Intelectual (Lei nº 9.279/1996) e respeitar a privacidade e os direitos pessoais.

A indústria de robótica tem experimentado um crescimento significativo nos últimos anos, impulsionado pelo avanço das tecnologias de automação e pela demanda crescente por soluções inteligentes em diversos setores. Dentro desse cenário, os carrinhos segue-linha se destacam como uma aplicação prática e educativa da robótica, especialmente popular entre entusiastas de tecnologia e em ambientes educacionais. No entanto, o mercado atual para carrinhos segue-linha é caracterizado por um alto custo de produtos similares, que muitas vezes limita a acessibilidade para uma base mais ampla de consumidores. Além disso, o número reduzido de empresas concorrentes pode restringir a variedade e inovação de produtos disponíveis.

Segundo Makedon, Mykhailenko e Vazov (2021), "o nível de automação na indústria automotiva é geralmente muito mais alto do que em todos os outros setores. Desde 2014, um número significativo de robôs industriais foi entregue às empresas da indústria automobilística sul-coreana. [...] De acordo com especialistas da IFR, projetos para a produção de baterias para carros híbridos e veículos elétricos podem ser a fonte de um aumento significativo na densidade de robotização"[MAKEDON; MYKHAILENKO; VAZOV, 2021].

Essas tendências globais destacam a importância da automação em vários setores, inclusive na indústria de carrinhos segue-linha, que pode se beneficiar da integração de novas tecnologias para aprimorar a eficiência e a acessibilidade.

O Carrinho Seguidor de Linha é um exemplo fascinante de como a robótica pode ser aplicada para melhorar processos em diversos setores. Na manufatura, sua capacidade de transportar componentes de forma autônoma não apenas otimiza o fluxo de trabalho, mas também pode contribuir para um ambiente de trabalho mais seguro, minimizando a necessidade de transporte manual de itens, o que pode levar a lesões. Na agricultura, a automação de tarefas como o transporte de insumos pode aumentar a produtividade e permitir que os agricultores se concentrem em aspectos mais estratégicos da produção. Além disso, a tecnologia subjacente ao Carrinho Seguidor de Linha oferece insights valiosos para o avanço de veículos autônomos, que têm o potencial de transformar não apenas a logística interna das indústrias, mas também o transporte urbano e rural. A integração de sistemas de navegação mais avançados e a capacidade de adaptação a ambientes mais complexos são desenvolvimentos que podemos esperar como resultado direto do aprimoramento contínuo dessas tecnologias. Assim, o Carrinho Seguidor de Linha não é apenas uma ferramenta útil no presente, mas também um precursor vital para as inovações futuras na robótica e na automação.

%\textcolor{red}{Esta seção terá no máximo duas páginas para apresentar ao leitor uma breve e atualizada revisão bibliográfica sobre o tema do projeto. A Introdução mostrará ao leitor “como está o mundo atual em relação produto desenvolvido”, ou seja, citará algumas pesquisas com produtos similares, publicadas em Journals ou Teses de Doutorado.}

% \begin{equation}
%     \mathbf{F} = m \mathbf{a}
% \end{equation}

% \begin{equation}
%     x_{1,2} = \frac{-b \pm \sqrt{b^2-4ac}}{2a}
% \end{equation}

% Repare que $b^2-4ac$ pode ser negativo, gerando raízes complexas.

%\textcolor{red}{Além de pesquisas científicas, é essencial que as principais legislações sobre o tema do projeto sejam citadas para atualizar o leitor. Se o grupo ou o professor orientador julgarem relevante, indicadores de mercado devem ser adicionados, como alto custo de produtos similares, baixo número de empresas concorrentes ou número estimado de consumidores finais.}

%\textcolor{red}{A Introdução finalizará com 1 (um) parágrafo de justificativa. Nesse parágrafo, o grupo ressaltará o motivo para determinar que o produto proposto atenda às necessidades atual do mercado/consumidor final ou melhora algum sistema já existente, por exemplo, composto por materiais reciclados.}

%\textcolor{red}{Todas as Tabelas e Figuras devem ser referenciadas ao longo do texto, já que elas são ferramentas para auxiliar no entendimento do mesmo. Use o comando ``\textsf{\textbackslash label\{\}}'' junto a Tabelas e Figuras para referencia-las. A Fig. \ref{fig:exemplo_fig} exemplifica como adicionar imagens ao texto.}

%\textcolor{red}{Para citarem trabalhos, utilizem o comando ``\textsf{\textbackslash cite\{\}}''. Evitem ao máximo o uso de outros comandos, tais como o ``\textsf{\textbackslash citeonline\{\}}''.}

\begin{figure}[htpb]
\centering
\includegraphics[width=\textwidth]{figuras/fga.png}
\caption{\textcolor{red}{Exemplo de figura adicionada em \LaTeX.}}
\label{fig:exemplo_fig}
\end{figure}

% \chapter*[Introdução]{Introdução}
% \addcontentsline{toc}{chapter}{Introdução}

% Este documento apresenta considerações gerais e preliminares relacionadas 
% à redação de relatórios de Projeto de Graduação da Faculdade UnB Gama 
% (FGA). São abordados os diferentes aspectos sobre a estrutura do trabalho, 
% uso de programas de auxilio a edição, tiragem de cópias, encadernação, etc.

% Este template é uma adaptação do ABNTeX2\footnote{\url{https://github.com/abntex/abntex2/}}.
