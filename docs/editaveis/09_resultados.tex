\chapter{Resultados experimentais}

\section{Características gerais}

\textcolor{red}{Introduzir os principais pontos deste capítulo. Cada experimento deve conter explicações completas que garantam sua repetibilidade:
\begin{itemize}
    \item Hipóteses levantadas
    \item Condições de contorno
    \item Resultados esperados
    \item Materiais e métodos
    \item Precisão e acurácia das medidas obtidas.
\end{itemize}
}

\section{Experimentos da estrutura}

\textcolor{red}{Apresentar com detalhes os experimentos feitos para conferir se os objetivos da estrutura foram atendidos.}

\section{Experimentos de \textit{hardware}}

\textcolor{red}{Apresentar com detalhes os experimentos feitos para conferir se os objetivos do \textit{hardware} foram atendidos.}

\section{Experimentos de consumo energético}

\textcolor{red}{Apresentar com detalhes os experimentos feitos para conferir se as demandas energéticas do projeto foram atendidas.}

\section{Experimentos de \textit{software}}

\textcolor{red}{Apresentar com detalhes os experimentos feitos para conferir se os objetivos do \textit{software} foram atendidos.}

\section{Experimentos de integração}

\textcolor{red}{Apresentar com detalhes os experimentos feitos para conferir se os objetivos do projeto como um todo foram atendidos.}

%\textcolor{red}{Com relação ao \textit{software}, será necessário apresentar pacotes de componentes de \textit{software}, suas funções e características, e explicar as decisões de projeto.}

