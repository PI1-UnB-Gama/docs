\chapter{Termo de Abertura do Projeto}

\textcolor{red}{Termo de abertura do projeto / Project Charter. Um documento publicado pelo iniciador ou patrocinador do projeto que autoriza formalmente a existência de um projeto e fornece ao gerente do projeto a autoridade para aplicar os recursos organizacionais nas atividades do projeto.}

\section{Dados do projeto}
\begin{description}
    \item [Nome do Projeto:] Carrinho seguidor de linha
    \item [Data de abertura:] DD/MM/AAAA
    \item [Código:] 1-A
    \item [Patrocinador:] Universidade de Brasília
    % \item [Responsável:] ???
    \item [Gerente:] Sarah Loriato Nazareth Franco/190020008/sarahloriato2@gmail.com/(61)99911-1112
\end{description}

\section{Objetivos}
\textcolor{red} {O que a empresa pretende obter com a realização do projeto. Descrever o que se pretende realizar para resolver o problema central ou explorar a oportunidade identificada. Para a correta definição do objetivo siga a regra "SMART":
\begin{description}
    \item [\textit{Specific} (específico):] Deve ser redigido de forma clara, concisa e compreensiva;
    \item [\textit{Measurable} (mensurável):] O objetivo específico deve ser mensurável, ou seja, 
possível de ser medido por meio de um ou mais indicadores;
    \item [\textit{Agreed} (acordado):] Deve ser acordado com as partes interessadas, ou seja, as áreas envolvidas na empresa: P\&D, Produção, Comercial, Marketing, Financeira, Jurídica, Manutenção, ambiental, entre outras;
    \item [\textit{Realistic} (realista):] Deve estar centrado na realidade, no que é possível de ser feito considerando as premissas e restrições existentes, como: orçamento e tempo;
    \item [\textit{Time Bound} (Limitado no tempo):] Deve ter um prazo determinado para sua finalização.
\end{description}
}

\section{Mercado-alvo}

\textcolor{red}{Pessoas, empresas, instituições etc. que usufruirão dos produtos, serviços e resultados gerados pelo projeto, cujos requisitos (tópico abaixo) devem atender as suas necessidades. Podem ser internas ou externas à organização, mas, merecem destaque especial, pois, o projeto está sendo feito para atendê-los de forma direta ou indireta.}

\section{Requisitos}

\textcolor{red}{Listar os fatos que são essenciais para o consumidor adquirir o produto, exemplo: cor, tamanho, material, tempo de vida-útil, dispositivo de segurança \textbf{x}, entre outros.}

\section{Justificativa}

\textcolor{red}{Informar o problema ou a oportunidade (necessidade) que justifica o porquê de o projeto ser realizado. Por exemplo: atende uma demanda específica do consumidor final; supre uma necessidade do mercado comercializador; é um diferencial X para o órgão regulamentador.}

\section{Indicadores}

\textcolor{red}{Listar até 10 indicadores que determinam o mercado consumidor do produto desenvolvido: exemplo: 1) n° de alunos da FGA que utilizam ônibus às 18:00; 2) n° de usuários do restaurante universitários, 3) número de idosos classificados como público-alvo no DF e no estado de Goiás, 4) n° de empresas de segurança registradas no DF etc.}
% \section{Aprovações}

% \begin{tabular}{ l l }
%   \textbf{Patrocinador:} & \_\_\_\_\_\_\_\_\_\_\_\_\_\_\_\_\_\_\_\_\_\_\_\_\_\_\_\_\_\_\_\_\_\_ \\
%   & \\
%   \textbf{CEO:} & \_\_\_\_\_\_\_\_\_\_\_\_\_\_\_\_\_\_\_\_\_\_\_\_\_\_\_\_\_\_\_\_\_\_ \\
%   & \\
%   \textbf{Gerente:} & \_\_\_\_\_\_\_\_\_\_\_\_\_\_\_\_\_\_\_\_\_\_\_\_\_\_\_\_\_\_\_\_\_\_ \\
% \end{tabular}
