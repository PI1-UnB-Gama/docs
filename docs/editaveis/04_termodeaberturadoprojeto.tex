\chapter{Termo de Abertura do Projeto}

%\textcolor{red}{Termo de abertura do projeto / Project Charter. Um documento publicado pelo iniciador ou patrocinador do projeto que autoriza formalmente a existência de um projeto e fornece ao gerente do projeto a autoridade para aplicar os recursos organizacionais nas atividades do projeto.}

\section{Dados do projeto}
\begin{description}
    \item [Nome do Projeto:] Carrinho seguidor de linha
    \item [Data de abertura:] 17/04/2024
    \item [Código:] 1-A
    \item [Patrocinador:] Universidade de Brasília
    % \item [Responsável:] ???
    \item [Gerente:] Sarah Loriato Nazareth Franco/190020008/sarahloriato2@gmail.com/(61)99911-1112
\end{description}

\section{Objetivos}
O projeto visa o desenvolvimento de um carrinho totalmente autônomo capaz de seguir trilhas marcadas no chão, transportando um ovo de galinha sem danificá-lo ao longo do percurso. Os principais objetivos são:

\begin{description}
    \item [\textit{Specific} (específico):] Desenvolver um carrinho que seja capaz de percorrer completamente três trilhas marcadas no chão, sem auxílio externo para a sua movimentação e início de trajeto, exceto para ser iniciado.
    \item [\textit{Measurable} (mensurável):] O sucesso do carrinho será medido através de sua habilidade de completar os trajetos no menor tempo possível, sem causar dano ao ovo de galinha transportado. O registro de dados como trajetória percorrida, velocidade instantânea, aceleração instantânea, tempo de percurso e consumo energético ajudará a mensurar a performance.
    \item [\textit{Agreed} (acordado):] O projeto será acordado com as partes interessadas dentro da universidade, incluindo professores e alunos das diferentes engenharias da FGA, garantindo uma abordagem multidisciplinar no desenvolvimento do carrinho.
    \item [\textit{Realistic} (realista):] O objetivo é baseado na realidade do que pode ser alcançado pelos estudantes, utilizando conhecimentos de todas as engenharias da FGA, respeitando as restrições de não utilização de sistemas prontos do mercado e focando na originalidade do projeto.
    \item [\textit{Time Bound} (Limitado no tempo):] O projeto tem um prazo determinado para sua finalização, alinhado com o calendário acadêmico e as datas estipuladas para os pontos de controle e apresentação final.
\end{description}


\section{Mercado-alvo}  

Empresas de logística e transporte, particularmente empresas de comércio atacadista e armazéns no DF e região. 

Empresas de fazendas verticais, empresas que visam produzir alimentos de forma eficiente e automatizada em grandes centros urbanos. 

\section{Requisitos} 

Para o setor de armazenamento, o produto requer estrutura reforçada para grandes volumes e cargas, com dispositivos de segurança integrados, com durabilidade, além de fácil uso e manutenção.  

Para o setor de agricultura vertical, necessita eficiência energética, precisão de navegação, integração com sistemas de controle e automação para irrigação entre outras funções como identificação e controle de pragas. 

Fora os requisitos para a indústria, podemos citar:
\begin{enumerate}
    \item Versatilidade de terreno, se pode ser usado em diferentes superfícies.
    \item Eficiência em percorrer distâncias em um tempo razoável.
    \item Funcionalidades de monitoramento através do software.
    \item Durabilidade e qualidade dos materiais.
    \item Segurança do ovo durante o percurso.
    \item Design ergonômico e confortável.
    \item Tempo de vida-útil.
    \item Facilidade de uso.
    \item Amortecimento.
    \item Tamanho.
\end{enumerate}

\section{Justificativa} 

\textcolor{black}{
Como justificativa para o desenvolvimento deste trabalho, há de se pensar em como a robótica e a automação já representam uma das formas mais eficientes e tecnológicas presentes em comércios e indústrias. Com isso em mente, realizar um projeto onde criaremos um carrinho seguidor de linha e que consegue carregar algo, pode nos dar uma dimensão melhor em como aplicar isso na indústria.}

{Transporte e distribuição/organização de produtos em grandes volumes os armazéns podem se beneficiar de um carro seguidor de linha para otimizar as operações de movimentação de mercadorias. 

Inovação significativa no setor agrícola, trazendo uma série de benefícios e oportunidades ainda em meio urbano. Automatizar tarefas agrícolas com precisão e qualidade, como plantio, irrigação e colheita, monitoramento e controle pode resultar em uma redução significativa nos custos operacionais, contribuindo ainda para a sustentabilidade ambiental }

{O carrinho seguidor de linha em meios logísticos se apresenta como uma ótima solução, pois além de dispensar a mão de obra de um organizador, prioriza a eficiência por meio de algoritmos que estão sempre em desenvolvimento. Para que esse produto chegue nesse nível de qualidade a integração entre diversas engenharias desempenha um papel um fundamental para sua construção e otimização, trabalho esse será feito pela equipe de Projeto de Integrador de Engenharias.
}

\section{ Indicadores} 

1) Região geográfica onde os consumidores estão concentrados, sendo 66 empresas de comércio atacadista no DF. 

2) 964 supermercados ativos. 

3) 14 escolas técnicas. 

4) Empresas de armazenamento, carga e descarga no Distrito Federal totalizam 446. 

5) 52 hospitais (rede pública e privada). 

6) Existem mais de 20 fazendas verticais ou urbanas no Brasil, sendo a maior da América Latina em São Paulo. 

7) 5,25 mil estabelecimentos agropecuários. 

8) 5,61 mil empresas industriais. 

9) 446 empresas de armazenamento.

%\textcolor{red}{Pessoas, empresas, instituições etc. que usufruirão dos produtos, serviços e resultados gerados pelo projeto, cujos requisitos (tópico abaixo) devem atender as suas necessidades. Podem ser internas ou externas à organização, mas, merecem destaque especial, pois, o projeto está sendo feito para atendê-los de forma direta ou indireta.}


% \textcolor{red}{Listar até 10 indicadores que determinam o mercado consumidor do produto desenvolvido: exemplo: 1) n° de alunos da FGA que utilizam ônibus às 18:00; 2) n° de usuários do restaurante universitários, 3) número de idosos classificados como público-alvo no DF e no estado de Goiás, 4) n° de empresas de segurança registradas no DF etc.}
% \section{Aprovações}

% \begin{tabular}{ l l }
%   \textbf{Patrocinador:} & \_\_\_\_\_\_\_\_\_\_\_\_\_\_\_\_\_\_\_\_\_\_\_\_\_\_\_\_\_\_\_\_\_\_ \\
%   & \\
%   \textbf{CEO:} & \_\_\_\_\_\_\_\_\_\_\_\_\_\_\_\_\_\_\_\_\_\_\_\_\_\_\_\_\_\_\_\_\_\_ \\
%   & \\
%   \textbf{Gerente:} & \_\_\_\_\_\_\_\_\_\_\_\_\_\_\_\_\_\_\_\_\_\_\_\_\_\_\_\_\_\_\_\_\_\_ \\
% \end{tabular}
