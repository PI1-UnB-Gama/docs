\documentclass[
  12pt, % Tamaho da fonte
  openright, % Capítulos começam em página ímpar
  twoside, % Impressão em frente e verso
  a4paper, % Tamanho do papel
  english, % Idioma adicional para hifenização
  french, % Idioma adicional para hifenização
  spanish, % Idioma adicional para hifenização
  brazil % O último idioma é o principal do documento
]{abntex2}

%
% Pacotes
%
\usepackage{lmodern} % Usa a fonte Latin Modern
\usepackage[T1]{fontenc} % Seleção de códigos de fonte
\usepackage[utf8]{inputenc}	% Codificação do documento
\usepackage{indentfirst} % Indenta o primeiro parágrafo de cada seção
\usepackage{color} % Controle das cores
\usepackage{graphicx} % Inclusão de gráficos
\usepackage{microtype} % Para melhorias de justificação
\usepackage{pdfpages} % Para inclusão de arquivos PDF

%
% Pacotes de citações
%
\usepackage[brazilian,hyperpageref]{backref} % Páginas com as citações
\usepackage{abntex2cite} % Citações no padrão ABNT

%
% Configurações de pacotes
%
\renewcommand{\backrefpagesname}{Citado na(s) página(s):~}
\renewcommand{\backref}{}
\renewcommand*{\backrefalt}[4]{
	\ifcase #1
		Nenhuma citação no texto.
	\or
		Citado na página #2.
	\else
		Citado #1 vezes nas páginas #2.%
	\fi
}

%
% Informações de dados para a capa e folha de rosto
%
\titulo{Projeto Integrador de Engenharia 1}
\autor{Grupo 01 (T01)}
\local{Brasília, DF}
\data{2024, v0.2.0}
\instituicao{Universidade de Brasília \par Faculdade do Gama}
\tipotrabalho{Relatório técnico}
% O preâmbulo deve conter o tipo do trabalho, o objetivo e o nome da instituição
\preambulo{
  Trabalho submetido à disciplina de Projeto Integrador de Engenharia 1 da
  Universidade de Brasília, ministrada pelo professor Diogo Garcia.
}
\orientador{Diogo Garcia}

% Informações do PDF
\makeatletter
\hypersetup{
  pdftitle={\@title},
  pdfauthor={\@author},
  pdfsubject={\imprimirpreambulo},
  pdfcreator={LaTeX with abnTeX2},
  pdfkeywords={abnt}{latex}{abntex}{projeto integrador de engenharia 1},
  colorlinks=true, % false: links em caixas; true: links coloridos
  linkcolor=blue, % Cor dos links internos
  citecolor=blue, % Cor dos links de citação
  filecolor=magenta, % Cor dos links para arquivos
  urlcolor=blue,
  bookmarksdepth=4
}
\makeatother

%
% Configurações de aparência do PDF final
%
\definecolor{blue}{RGB}{41,5,195} % Cor azul para os links

% Posiciona figuras e tabelas no topo da página quando adicionadas sozinhas
% em um página em branco. Ver https://github.com/abntex/abntex2/issues/170
\makeatletter
\setlength{\@fptop}{5pt}
\makeatother

% Possibilita criação de Quadros e Lista de quadros.
% Ver https://github.com/abntex/abntex2/issues/176
\newcommand{\quadroname}{Quadro}
\newcommand{\listofquadrosname}{Lista de quadros}

\newfloat[chapter]{quadro}{loq}{\quadroname}
\newlistof{listofquadros}{loq}{\listofquadrosname}
\newlistentry{quadro}{loq}{0}

% configurações para atender às regras da ABNT
\setfloatadjustment{quadro}{\centering}
\counterwithout{quadro}{chapter}
\renewcommand{\cftquadroname}{\quadroname\space}
\renewcommand*{\cftquadroaftersnum}{\hfill--\hfill}

% Ver https://github.com/abntex/abntex2/issues/176
\setfloatlocations{quadro}{hbtp}

%
% Espaçamentos entre linhas e parágrafos
%
\setlength{\parindent}{1.3cm}
\setlength{\parskip}{0.2cm}

%
% Início do documento
%
\begin{document}

\selectlanguage{brazil}
\frenchspacing

%
% Elementos pré-textuais
%
\pretextual

%
% Capa
%
\imprimircapa

%
% Folha de rosto
% (o * indica que haverá a ficha bibliográfica)
%
\imprimirfolhaderosto*

%
% TODO: Inserir ficha catalográfica
%

%
% Inserir folha de aprovação
%

% Isto é um exemplo de Folha de aprovação, elemento obrigatório da NBR
% 14724/2011 (seção 4.2.1.3). Você pode utilizar este modelo até a aprovação
% do trabalho. Após isso, substitua todo o conteúdo deste arquivo por uma
% imagem da página assinada pela banca com o comando abaixo:
%
% \begin{folhadeaprovacao}
%   \includepdf{folhadeaprovacao_final.pdf}
% \end{folhadeaprovacao}
%
\begin{folhadeaprovacao}
  \begin{center}
    {\ABNTEXchapterfont\large\imprimirautor}

    \vspace*{\fill}\vspace*{\fill}
    \begin{center}
      \ABNTEXchapterfont\bfseries\Large\imprimirtitulo
    \end{center}
    \vspace*{\fill}

    \hspace{.45\textwidth}
    \begin{minipage}{.5\textwidth}
        \imprimirpreambulo
    \end{minipage}%
    \vspace*{\fill}
   \end{center}

   Trabalho aprovado. \imprimirlocal, 24 de abril de 2024:

   \assinatura{\textbf{\imprimirorientador} \\ Orientador}
   \assinatura{\textbf{Rafael Rodrigues} \\ Coorientador}
   \assinatura{\textbf{Jungpyo Lee} \\ Coorientador}
   \assinatura{\textbf{Juliana Petrocchi} \\ Coorientador}
   \assinatura{\textbf{Ricardo Ajax} \\ Coorientador}

   \begin{center}
    \vspace*{0.5cm}
    {\large\imprimirlocal}
    \par
    {\large\imprimirdata}
    \vspace*{1cm}
  \end{center}
\end{folhadeaprovacao}

%
% Dedicatória
%
\begin{dedicatoria}
  \vspace*{\fill}
  \centering
  \noindent
  \textit{
    Dedicamos este trabalho aos nossos orientadores, que durante o semestre nos
    proporcionaram a melhor experiência para escrever um trabalho de qualidade.
  }
  \vspace*{\fill}
\end{dedicatoria}

%
% TODO: Agradecimentos
%

%
% Resumo
%
% Ajusta o espaçamento dos parágrafos do resumo
\setlength{\absparsep}{18pt}
\begin{resumo}
  O seguidor de linha, também chamado de \textit{follow line} ou
  \textit{line follower robot}, é uma categoria de robôs autônomos geralmente
  semelhantes a carros de corrida cujo objetivo é realizar um percurso no menor
  tempo possível seguindo uma linha preta no chão. Este trabalho apresenta o
  projeto de um seguidor de linha que utiliza circuitos eletrônicos, sensores e
  softwares para realizar três percursos aleatórios. Além de percorrer os
  trajetos, o robô também deve ser capaz de carregar um ovo sem quebrá-lo, e
  apresentar dados de telemetria em algum dispositivo externo. O projeto
  foi desenvolvido por estudantes de engenharia da Universidade de Brasília
  durante a disciplina de Projeto Integrador de Engenharia 1, ministrada pelo
  professor Diogo Garcia.

  \textbf{Palavras-chave}:
    Seguidor de linha. Carrinho seguidor de linha. Engenharia. Projeto
    Integrador de Engenharia 1. Universidade de Brasília. Faculdade do Gama.
    Diogo Garcia.
\end{resumo}


%
% Inserir lista de ilustrações
%
\pdfbookmark[0]{\listfigurename}{lof}
\listoffigures*
\cleardoublepage

%
% Inserir lista de quadros
%
\pdfbookmark[0]{\listofquadrosname}{loq}
\listofquadros*
\cleardoublepage

%
% Inserir lista de tabelas
%
\pdfbookmark[0]{\listtablename}{lot}
\listoftables*
\cleardoublepage

%
% inserir lista de abreviaturas e siglas
%
\begin{siglas}
  \item[FGA] Faculdade do Gama
  \item[INATEL] Instituto Nacional de Telecomunicações
  \item[ANATEL] Agência Nacional de Telecomunicações
  \item[IFR] \textit{International Federation of Robotics}
\end{siglas}

%
% Inserir lista de símbolos
%
\begin{simbolos}
  \item[$ \Lambda $] Lambda (maiúsculo)
\end{simbolos}

%
% Inserir sumário
%
\pdfbookmark[0]{\contentsname}{toc}
\tableofcontents*
\cleardoublepage

\textual

\chapter{Introdução}

Um seguidor de linha é um robô autônomo que segue uma linha preta no chão. O
objetivo é percorrer um percurso no menor tempo possível, evitando obstáculos e
mantendo-se dentro dos limites da pista. De acordo com a INATEL, a definição de
um seguidor de linha é:

\begin{citacao}
  \lbrack...\rbrack 
  \space
  uma categoria de robôs autônomos que, na maioria das vezes, se assemelham a
  carros de corrida. Através da combinação de motores, sensores e inteligência
  artificial, o principal objetivo é percorrer um determinado percurso no menor
  tempo possível. Os métodos utilizados, como mapeamento e a diferenciação entre
  retas e curvas, para que a estrutura mecânica tenha o melhor aproveitamento,
  podem ser empregados em diversos métodos de controle.
  \cite{INATEL:Seguidor-de-Linha}
\end{citacao}

Essa definição destaca a importância da integração de diferentes conceitos e
tecnologias para o desenvolvimento de um seguidor de linha eficiente. A
combinação de motores, sensores e algoritmos de controle é essencial para
garantir que o robô possa navegar com precisão e rapidez, evitando colisões e
desvios da linha de referência. Além disso, a capacidade de diferenciar entre
retas e curvas e de realizar mapeamentos do ambiente são aspectos fundamentais
para o desempenho do robô em diferentes cenários.

Uma vez que os robôs têm, frequentemente, um propósito de competição, a diretriz
que o projeto deve seguir varia da competição ou propósito final. No Brasil, não
existem quaisquer proibições sobre pequenos robôs seguidores de linha, as
possíveis legislações que podem abranger esse robô variam de acordo com seu
propósito, como se ele irá ou não ser comercializado \cite{Lei:8078:1990}, se
ele utiliza ou não formas sem fio para sua partida, de acordo com as
conformidades da ANATEL \cite{ANATEL:Manual-de-Orientacoes}, e se possuindo uma
câmera, ela faz registro e armazenamento das gravações \cite{Lei:12651:2012}.
Por fim, é responsabilidade dos construtores do robô não infringir leis como a
Lei de Propriedade Intelectual \cite{Lei:9279:1996} e respeitar a privacidade e
os direitos pessoais.

A indústria de robótica tem experimentado um crescimento significativo nos
últimos anos, impulsionado pelo avanço das tecnologias de automação e pela
demanda crescente por soluções inteligentes em diversos setores. Dentro desse
cenário, os seguidores de linha se destacam como uma aplicação prática e
educativa da robótica, especialmente popular entre entusiastas de tecnologia e
em ambientes educacionais. No entanto, o mercado atual para carrinhos
seguidor de linha é caracterizado por um alto custo de produtos similares, que
muitas vezes limita a acessibilidade para uma base mais ampla de consumidores.
Além disso, o número reduzido de empresas concorrentes pode restringir a
variedade e inovação de produtos disponíveis.

% Pular para a próxima página. Segundo as normas da ABNT,
% parágrafos não devem ser divididos entre páginas.
\clearpage

% TODO: Colocar referência correta do artigo citado abaixo.
Segundo Makedon, Mykhailenko e Vazov:

\begin{citacao}
  \lbrack...\rbrack 
  \space
  o nível de automação na indústria automotiva é geralmente muito mais alto do
  que em todos os outros setores. Desde 2014, um número significativo de robôs
  industriais foi entregue às empresas da indústria automobilística sul-coreana.
  \lbrack...\rbrack\space De acordo com especialistas da IFR, projetos para a
  produção de baterias para carros híbridos e veículos elétricos podem ser a
  fonte de um aumento significativo na densidade de robotização.
  \cite[p.~6]{Makedon:Dominants-and-Features-Market-Robotics}

\end{citacao}

Essas tendências globais destacam a importância da automação em vários setores,
inclusive na indústria de seguidores de linha, que pode se beneficiar da
integração de novas tecnologias para aprimorar a eficiência e a acessibilidade.

Neste trabalho, propomos o desenvolvimento de um carrinho seguidor de linha
autônomo, que seja capaz de percorrer trajetos pré-definidos sem auxílio
externo, além de transportar um ovo sem quebrá-lo, e enviar dados de telemetria,
como o trajeto percorrido, a velocidade instantânea, a aceleração instantânea, o
consumo energético e o tempo de percurso, para um dispositivo externo ao robô.

\chapter{Termo de Abertura do Projeto}

\section{Dados do Projeto}

\begin{center}
  \begin{tabular}{|l|l|}
    \hline
    \textbf{Nome do Projeto} & Carrinho seguidor de linha \\
    \hline
    \textbf{Data de Abertura} & 17/04/2024 \\
    \hline
    \textbf{Código} & 1-A \\
    \hline
    \textbf{Patrocinador} & Universidade de Brasília \\
    \hline
    \textbf{Gerente} & Sarah Loriato Nazareth Franco (190020008) \\
    \hline
  \end{tabular}
\end{center}

\section{Objetivos}

Os objetivos do projeto são desenvolver um carrinho seguidor de linha autônomo
que seja capaz de percorrer trajetos pré-definidos sem auxílio externo, além de
transportar um ovo sem quebrá-lo. É possível descrever os objetivos do projeto
utilizando o acrônimo \textit{SMART} \cite{SMART-Goals:2017}, que seriam os
seguintes:

\subsection{\textit{Specific} (específico)}

Desenvolver um carrinho que seja capaz de percorrer completamente três
trilhas marcadas no chão, sem auxílio externo para a sua movimentação e
início de trajeto, exceto para ser iniciado.
 
\subsection{\textit{Measurable} (mensurável)}

O sucesso do carrinho será medido através de sua habilidade de completar os
trajetos no menor tempo possível, sem causar dano ao ovo transportado. O
registro de dados como trajetória percorrida, velocidade instantânea,
aceleração instantânea, tempo de percurso e consumo energético ajudará a
mensurar a performance.

\subsection{\textit{Agreed} (acordado)}

O projeto será acordado com as partes interessadas dentro da universidade,
incluindo professores e alunos das diferentes engenharias da FGA, garantindo
uma abordagem multidisciplinar no desenvolvimento do carrinho.

\subsection{\textit{Realistic} (realista)}

O projeto será baseado na realidade do que pode ser alcançado pelos estudantes,
utilizando conhecimentos de todas as engenharias da FGA, respeitando as
restrições impostas pelos professores e pela universidade.

\subsection{\textit{Time-bound} (limitado ao tempo)}

O projeto terá um prazo determinado para sua finalização, alinhado com o
calendário acadêmico e as datas estipuladas para os pontos de controle e
apresentação final.

\section{Mercado-alvo}

O carrinho seguidor de linha é um projeto que visa atender a demanda de
estudantes de engenharia que desejam desenvolver habilidades práticas e
multidisciplinares. Apesar disso, o projeto também pode ser utilizado por
empresas de logística e transporte, particularmente empresas de comércio
atacadista e armazéns no DF e região. Empresas de fazendas verticais, empresas
que visam produzir alimentos de forma eficiente e automatizada em grandes
centros urbanos também podem se beneficiar do projeto.

\chapter{Equipe de Trabalho}

\begin{quadro}[htb]
  \caption{\label{quadro_equipe_trabalho}Equipe de Trabalho}

  \begin{tabular}{|c|c|c|}
    \hline
    \textbf{Nome} & \textbf{Matrícula} & \textbf{Atribuição} \\
    \hline
    Sarah Loriato N. Franco & 190020008 & Gerente geral \\
    \hline
    Guilherme Elizeu de O. Freitas & 150128258 & Gerente de estrutura \\
    \hline
    Pedro Ambrosio A. Bianco & 231029805 & Gerente de qualidade e energia \\
    \hline
    Henrique Galdino Couto & 200058258 & Gerente de software \\
    \hline
    Rafael Silva Frota & 180067460 & Gerente de hardware \\
    \hline
    João Ricardo F. de Almeida & 190046562 & Colaborador de software \\
    \hline
    João Pedro Costa & 190030801 & Colaborador de software \\
    \hline
    Alexandre de Santana Beck & 211061350 & Colaborador de software \\
    \hline
    Rodrigo Braz F. Gontijo & 190116498 & Colaborador de software \\
    \hline
    Luis Felipe de S. Braga & 202016865 & Colaborador de software \\
    \hline
    Caio Alexandre O. Silva & 221007644 & Colaborador de software \\
    \hline
    Ana Caroline S. Batista & 202063186 & Colaboradora de estruturas \\
    \hline
    Maria Eliza Braga & 221022373 & Colaboradora de Estruturas \\
    \hline
  \end{tabular}

  \fonte{Elaborado pelos autores.}
\end{quadro}

\chapter{Projeto Conceitual do Produto}

\section{Características Gerais}

O projeto foi divido em quatro frentes de trabalho, sendo elas:
\textbf{estrutura}, \textbf{hardware}, \textbf{software} e \textbf{energia},
com o intuito de organizar estruturar de forma que os objetivos sejam alcançados
de forma eficiente, seguindo os requisitos estabelecidos pela equipe e pelos
orientadores. As atribuições das quatro frentes serão realizadas em paralelo, de
modo a garantir que a integração e o funcionamento das mesmas esteja de acordo
com o esperado inicialmente.

\subsection{Estrutura}

Será responsável pela definição, aquisição e montagem dos materiais necessários
para elaboração da estrutura do carrinho. Para alcançar os objetivos previstos,
a estrutura apresentada deverá sustentar e proteger um ovo de galinha durante a
execução dos trajetos determinados.

\subsection{Hardware}

Será responsável pela definição, aquisição e configuração dos componentes
eletrônicos que serão utilizados na elaboração do carrinho, além de garantir a
medição de dados em tempo real para trajetória percorrida, velocidade
instantânea, aceleração instantânea, tempo de percurso e consumo energético.

\subsection{Software}

Será responsável pela definição, documentação e programação dos códigos que
serão utilizados para definir o funcionamento do carrinho, além de realizar os
cálculos e armazenamento dos dados de trajetória percorrida, velocidade
instantânea, aceleração instantânea, tempo de percurso e consumo energético, e
enviar esses dados para um dispositivo externo para visualização.

\subsection{Energia}

Será responsável pela definição, aquisição e montagem dos componentes que
alimentarão o carrinho e seus demais componentes, além de efetuar o cálculo e
medições dos relacionados ao consumo energético do mesmo.


\chapter{Análise de Mercado}

\section{Mercado-alvo}

O seguidor de linha é um projeto que visa atender a demanda de estudantes de
engenharia que desejam desenvolver habilidades práticas e multidisciplinares.
Apesar disso, o projeto também pode ser utilizado por empresas de logística e
transporte, particularmente empresas de comércio atacadista e armazéns no
Distrito Federal e região. Empresas de fazendas verticais, empresas que visam
produzir alimentos de forma eficiente e automatizada em grandes centros urbanos
também podem se beneficiar do projeto.

\section{Requisitos}

Para o setor de armazenamento, o produto requer estrutura reforçada para grandes
volumes e cargas, com dispositivos de segurança integrados, com durabilidade,
além de fácil uso e manutenção. Para o setor de agricultura vertical, necessita
eficiência energética, precisão de navegação, integração com sistemas de
controle e automação para irrigação entre outras funções como identificação e
controle de pragas. 

Fora os requisitos para a indústria, podemos citar:

\begin{enumerate}
  \item versatilidade de terreno, se pode ser usado em diferentes superfícies;
  \item eficiência em percorrer distâncias em um tempo razoável;
  \item funcionalidades de monitoramento através do software;
  \item durabilidade e qualidade dos materiais;
  \item design ergonômico e confortável;
  \item tempo de vida-útil;
  \item facilidade de uso;
  \item amortecimento;
  \item tamanho.
\end{enumerate}

\section{Justificativa}

Como justificativa para o desenvolvimento deste trabalho, há de se pensar em
como a robótica e a automação já representam uma das formas mais eficientes e
tecnológicas presentes em comércios e indústrias. Com isso em mente, realizar um
projeto onde criaremos um seguidor de linha e que consiga carregar algo, pode
nos dar uma dimensão melhor em como aplicar isso na indústria.

Transporte e distribuição/organização de produtos em grandes volumes os armazéns
podem se beneficiar de um seguidor de linha para otimizar as operações de
movimentação de mercadorias. Inovação significativa no setor agrícola, trazendo
uma série de benefícios e oportunidades ainda em meio urbano. Automatizar
tarefas agrícolas com precisão e qualidade, como plantio, irrigação e colheita,
monitoramento e controle pode resultar em uma redução significativa nos custos
operacionais, contribuindo ainda para a sustentabilidade ambiental.

O seguidor de linha em meios logísticos se apresenta como uma ótima solução,
pois além de dispensar a mão de obra de um organizador, prioriza a eficiência
por meio de algoritmos que estão sempre em desenvolvimento. Para que esse
produto chegue nesse nível de qualidade a integração entre diversas engenharias
desempenha um papel um fundamental para sua construção e otimização.

\section{Indicadores}

Alguns indicadores que podem ser utilizados para avaliar o mercado-alvo, dentro
do Distrito Federal, são:

\begin{enumerate}
  \item 66 empresas de comércio atacadista;
  \item 964 supermercados ativos;
  \item 14 escolas técnicas;
  \item 446 empresas de armazenamento, carga e descarga;
  \item 52 hospitais (rede pública e privada);
  \item 5,25 mil estabelecimentos agropecuários;
  \item 5,61 mil empresas industriais;
  \item 446 empresas de armazenamento.
\end{enumerate}



%
% Elementos pós-textuais
%
\postextual

%
% Referências bibliográficas
%
\bibliography{refs}

\end{document}
