\documentclass[
  12pt, % Tamaho da fonte
  openright, % Capítulos começam em página ímpar
  twoside, % Impressão em frente e verso
  a4paper, % Tamanho do papel
  english, % Idioma adicional para hifenização
  french, % Idioma adicional para hifenização
  spanish, % Idioma adicional para hifenização
  brazil % O último idioma é o principal do documento
]{abntex2}

%
% Pacotes
%
\usepackage{lmodern} % Usa a fonte Latin Modern
\usepackage[T1]{fontenc} % Seleção de códigos de fonte
\usepackage[utf8]{inputenc}	% Codificação do documento
\usepackage{indentfirst} % Indenta o primeiro parágrafo de cada seção
\usepackage{color} % Controle das cores
\usepackage{graphicx} % Inclusão de gráficos
\usepackage{microtype} % Para melhorias de justificação
\usepackage{pdfpages} % Para inclusão de arquivos PDF

%
% Pacotes de citações
%
\usepackage[brazilian,hyperpageref]{backref} % Páginas com as citações
\usepackage[alf]{abntex2cite}	% Citações no padrão ABNT

%
% Configurações de pacotes
%
\renewcommand{\backrefpagesname}{Citado na(s) página(s):~}
\renewcommand{\backref}{}
\renewcommand*{\backrefalt}[4]{
	\ifcase #1
		Nenhuma citação no texto.
	\or
		Citado na página #2.
	\else
		Citado #1 vezes nas páginas #2.%
	\fi
}

%
% Informações de dados para a capa e folha de rosto
%
\titulo{Projeto Integrador de Engenharia 1}
\autor{Grupo 01 (T01)}
\local{Brasília, DF}
\data{2024, v0.2.0}
\instituicao{Universidade de Brasília \par Faculdade do Gama}
\tipotrabalho{Relatório técnico}
% O preâmbulo deve conter o tipo do trabalho, o objetivo e o nome da instituição
\preambulo{
  Trabalho submetido à disciplina de Projeto Integrador de Engenharia 1 da
  Universidade de Brasília, ministrada pelo professor Diogo Garcia.
}
\orientador{Diogo Garcia}

% Informações do PDF
\makeatletter
\hypersetup{
  pdftitle={\@title},
  pdfauthor={\@author},
  pdfsubject={\imprimirpreambulo},
  pdfcreator={LaTeX with abnTeX2},
  pdfkeywords={abnt}{latex}{abntex}{projeto integrador de engenharia 1},
  colorlinks=true, % false: links em caixas; true: links coloridos
  linkcolor=blue, % Cor dos links internos
  citecolor=blue, % Cor dos links de citação
  filecolor=magenta, % Cor dos links para arquivos
  urlcolor=blue,
  bookmarksdepth=4
}
\makeatother

%
% Configurações de aparência do PDF final
%
\definecolor{blue}{RGB}{41,5,195} % Cor azul para os links

% Posiciona figuras e tabelas no topo da página quando adicionadas sozinhas
% em um página em branco. Ver https://github.com/abntex/abntex2/issues/170
\makeatletter
\setlength{\@fptop}{5pt}
\makeatother

% Possibilita criação de Quadros e Lista de quadros.
% Ver https://github.com/abntex/abntex2/issues/176
\newcommand{\quadroname}{Quadro}
\newcommand{\listofquadrosname}{Lista de quadros}

\newfloat[chapter]{quadro}{loq}{\quadroname}
\newlistof{listofquadros}{loq}{\listofquadrosname}
\newlistentry{quadro}{loq}{0}

% configurações para atender às regras da ABNT
\setfloatadjustment{quadro}{\centering}
\counterwithout{quadro}{chapter}
\renewcommand{\cftquadroname}{\quadroname\space}
\renewcommand*{\cftquadroaftersnum}{\hfill--\hfill}

% Ver https://github.com/abntex/abntex2/issues/176
\setfloatlocations{quadro}{hbtp}

%
% Espaçamentos entre linhas e parágrafos
%
\setlength{\parindent}{1.3cm}
\setlength{\parskip}{0.2cm}

%
% Início do documento
%
\begin{document}

\selectlanguage{brazil}
\frenchspacing

%
% Elementos pré-textuais
%
\pretextual

%
% Capa
%
\imprimircapa

%
% Folha de rosto
% (o * indica que haverá a ficha bibliográfica)
%
\imprimirfolhaderosto*

%
% TODO: Inserir ficha catalográfica
%

%
% Inserir folha de aprovação
%

% Isto é um exemplo de Folha de aprovação, elemento obrigatório da NBR
% 14724/2011 (seção 4.2.1.3). Você pode utilizar este modelo até a aprovação
% do trabalho. Após isso, substitua todo o conteúdo deste arquivo por uma
% imagem da página assinada pela banca com o comando abaixo:
%
% \begin{folhadeaprovacao}
%   \includepdf{folhadeaprovacao_final.pdf}
% \end{folhadeaprovacao}
%
\begin{folhadeaprovacao}
  \begin{center}
    {\ABNTEXchapterfont\large\imprimirautor}

    \vspace*{\fill}\vspace*{\fill}
    \begin{center}
      \ABNTEXchapterfont\bfseries\Large\imprimirtitulo
    \end{center}
    \vspace*{\fill}

    \hspace{.45\textwidth}
    \begin{minipage}{.5\textwidth}
        \imprimirpreambulo
    \end{minipage}%
    \vspace*{\fill}
   \end{center}

   Trabalho aprovado. \imprimirlocal, 24 de abril de 2024:

   \assinatura{\textbf{\imprimirorientador} \\ Orientador}
   \assinatura{\textbf{Rafael Rodrigues} \\ Coorientador}
   \assinatura{\textbf{Jungpyo Lee} \\ Coorientador}
   \assinatura{\textbf{Juliana Petrocchi} \\ Coorientador}
   \assinatura{\textbf{Ricardo Ajax} \\ Coorientador}

   \begin{center}
    \vspace*{0.5cm}
    {\large\imprimirlocal}
    \par
    {\large\imprimirdata}
    \vspace*{1cm}
  \end{center}
\end{folhadeaprovacao}

%
% Dedicatória
%
\begin{dedicatoria}
  \vspace*{\fill}
  \centering
  \noindent
  \textit{
    Dedicamos este trabalho aos nossos orientadores, que durante o semestre nos
    proporcionaram a melhor experiência para escrever um trabalho de qualidade.
  }
  \vspace*{\fill}
\end{dedicatoria}

%
% TODO: Agradecimentos
%

%
% Resumo
%

% Ajusta o espaçamento dos parágrafos do resumo
\setlength{\absparsep}{18pt}
\begin{resumo}
  O seguidor de linha é uma categoria de robôs autônomos geralmente semelhantes
  a carros de corrida cujo objetivo é realizar um percurso no menor tempo
  possível. O carrinho seguidor de linha descrito nesse projeto vai um pouco
  além, além de ser capaz de realizar três percursos aleatórios no menor tempo
  possível, o carrinho precisa levar em si um ovo, que deve permanecer intacto
  ao final do percurso. Para isso, o projeto irá utilizar uma combinação de
  sensores, motores, algoritmos computacionais (incluindo inteligência
  artificial) e um hardwares de forma que o robô não saia da linha a ser
  disposta no chão e possa também calcular sua velocidade, distância percorrida
  e tempo de conclusão do percurso.

  \textbf{Palavras-chave}: Carrinho seguidor de linha. Engenharia. Projeto
    Integrador de Engenharia 1.
\end{resumo}

%
% Inserir lista de ilustrações
%
\pdfbookmark[0]{\listfigurename}{lof}
\listoffigures*
\cleardoublepage

%
% Inserir lista de quadros
%
\pdfbookmark[0]{\listofquadrosname}{loq}
\listofquadros*
\cleardoublepage

%
% Inserir lista de tabelas
%
\pdfbookmark[0]{\listtablename}{lot}
\listoftables*
\cleardoublepage

%
% inserir lista de abreviaturas e siglas
%
\begin{siglas}
  \item[DIY] Do It Yourself (Faça Você Mesmo)
  \item[ANATEL] Agência Nacional de Telecomunicações
  \item[FGA] Faculdade do Gama
\end{siglas}

%
% Inserir lista de símbolos
%
\begin{simbolos}
  \item[$ \Lambda $] Lambda (maiúsculo)
\end{simbolos}

%
% Inserir sumário
%
\pdfbookmark[0]{\contentsname}{toc}
\tableofcontents*
\cleardoublepage

\textual

\chapter{Introdução}

Os projetos de robôs seguidores de linha são uma ótima forma de introduzir tanto
grupos de estudantes ou entusiastas interessados em projetos DIY ao mundo da
engenharia, isto pois este é um projeto relativamente simples e que consegue
combinar diversos conceitos de engenharia, automação e programação. Além desses
conceitos, quando feito em grupo, o projeto tem a capacidade de ajudar no
desenvolvimento pessoal dos integrantes em áreas como trabalho em equipe,
resolução de problemas e também instiga o espírito competitivo dos integrantes.
Falando sobre espírito esportivo, atualmente existem diversas competições
estudantis tanto nacionais quanto internacionais onde são realizadas provas que
testam a capacidade autônoma do robô em paralelo com a capacidade do mesmo
realizar a prova no menor tempo possível. Mas essa não é a única aplicação desse
tipo de projeto, ele também possui um grande potencial de aplicações no mundo
real, onde os mesmos conceitos podem ser encontrados em robôs autônomos que são
utilizados em linhas de produção de grandes indústrias.

Uma vez que os robôs têm, frequentemente, um propósito de competição, a diretriz
que o projeto deve seguir varia da competição ou propósito final. No Brasil, não
existem quaisquer proibições sobre pequenos robôs seguidores de linha, as
possíveis legislações que podem abranger esse robô variam de acordo com seu
propósito, como se ele irá ou não ser comercializado \cite{Lei:8078:1990}, se
ele utiliza ou não formas sem fio para sua partida (conformidades da ANATEL) e,
se possuindo uma câmera, ela faz registro e armazenamento das gravações
\cite{Lei:12651:2012}. Por fim, é responsabilidade dos construtores do robô não
infringir leis como a Lei de Propriedade Intelectual \cite{Lei:9279:1996} e
respeitar a privacidade e os direitos pessoais.

A indústria de robótica tem experimentado um crescimento significativo nos
últimos anos, impulsionado pelo avanço das tecnologias de automação e pela
demanda crescente por soluções inteligentes em diversos setores. Dentro desse
cenário, os carrinhos segue-linha se destacam como uma aplicação prática e
educativa da robótica, especialmente popular entre entusiastas de tecnologia e
em ambientes educacionais. No entanto, o mercado atual para carrinhos
seguidor de linha é caracterizado por um alto custo de produtos similares, que
muitas vezes limita a acessibilidade para uma base mais ampla de consumidores.
Além disso, o número reduzido de empresas concorrentes pode restringir a
variedade e inovação de produtos disponíveis.

% Pular para a próxima página. Segundo as normas da ABNT,
% parágrafos não devem ser divididos entre páginas.
\clearpage

% TODO: Colocar referência correta do artigo citado abaixo.
Segundo Makedon, Mykhailenko e Vazov (2021), ``o nível de automação na indústria
automotiva é geralmente muito mais alto do que em todos os outros setores. Desde
2014, um número significativo de robôs industriais foi entregue às empresas da
indústria automobilística sul-coreana. [...] De acordo com especialistas da IFR,
projetos para a produção de baterias para carros híbridos e veículos elétricos
podem ser a fonte de um aumento significativo na densidade de robotização''
[MAKEDON; MYKHAILENKO; VAZOV, 2021]. Essas tendências globais destacam a
importância da automação em vários setores, inclusive na indústria de carrinhos
seguidor de linha, que pode se beneficiar da integração de novas tecnologias
para aprimorar a eficiência e a acessibilidade.

O carrinho seguidor de linha é um exemplo fascinante de como a robótica pode ser
aplicada para melhorar processos em diversos setores. Na manufatura, sua
capacidade de transportar componentes de forma autônoma não apenas otimiza o
fluxo de trabalho, mas também pode contribuir para um ambiente de trabalho mais
seguro, minimizando a necessidade de transporte manual de itens, o que pode
levar a lesões. Na agricultura, a automação de tarefas como o transporte de
insumos pode aumentar a produtividade e permitir que os agricultores se
concentrem em aspectos mais estratégicos da produção. Além disso, a tecnologia
subjacente ao carrinho seguidor de linha oferece insights valiosos para o avanço
de veículos autônomos, que têm o potencial de transformar não apenas a logística
interna das indústrias, mas também o transporte urbano e rural. A integração de
sistemas de navegação mais avançados e a capacidade de adaptação a ambientes
mais complexos são desenvolvimentos que podemos esperar como resultado direto do
aprimoramento contínuo dessas tecnologias. Assim, o carrinho seguidor de linha
não é apenas uma ferramenta útil no presente, mas também um precursor vital para
as inovações futuras na robótica e na automação.

\chapter{Termo de Abertura do Projeto}

\section{Dados do Projeto}

\begin{center}
  \begin{tabular}{|l|l|}
    \hline
    \textbf{Nome do Projeto} & Carrinho seguidor de linha \\
    \hline
    \textbf{Data de Abertura} & 17/04/2024 \\
    \hline
    \textbf{Código} & 1-A \\
    \hline
    \textbf{Patrocinador} & Universidade de Brasília \\
    \hline
    \textbf{Gerente} & Sarah Loriato Nazareth Franco (190020008) \\
    \hline
  \end{tabular}
\end{center}

\section{Objetivos}

Os objetivos do projeto são desenvolver um carrinho seguidor de linha autônomo
que seja capaz de percorrer trajetos pré-definidos sem auxílio externo, além de
transportar um ovo sem quebrá-lo. É possível descrever os objetivos do projeto
utilizando o acrônimo \textit{SMART} \cite{SMART-Goals:2017}, que seriam os
seguintes:

\subsection{\textit{Specific} (específico)}

Desenvolver um carrinho que seja capaz de percorrer completamente três
trilhas marcadas no chão, sem auxílio externo para a sua movimentação e
início de trajeto, exceto para ser iniciado.
 
\subsection{\textit{Measurable} (mensurável)}

O sucesso do carrinho será medido através de sua habilidade de completar os
trajetos no menor tempo possível, sem causar dano ao ovo transportado. O
registro de dados como trajetória percorrida, velocidade instantânea,
aceleração instantânea, tempo de percurso e consumo energético ajudará a
mensurar a performance.

\subsection{\textit{Agreed} (acordado)}

O projeto será acordado com as partes interessadas dentro da universidade,
incluindo professores e alunos das diferentes engenharias da FGA, garantindo
uma abordagem multidisciplinar no desenvolvimento do carrinho.

\subsection{\textit{Realistic} (realista)}

O projeto será baseado na realidade do que pode ser alcançado pelos estudantes,
utilizando conhecimentos de todas as engenharias da FGA, respeitando as
restrições impostas pelos professores e pela universidade.

\subsection{\textit{Time-bound} (limitado ao tempo)}

O projeto terá um prazo determinado para sua finalização, alinhado com o
calendário acadêmico e as datas estipuladas para os pontos de controle e
apresentação final.

\section{Mercado-alvo}

O carrinho seguidor de linha é um projeto que visa atender a demanda de
estudantes de engenharia que desejam desenvolver habilidades práticas e
multidisciplinares. Apesar disso, o projeto também pode ser utilizado por
empresas de logística e transporte, particularmente empresas de comércio
atacadista e armazéns no DF e região. Empresas de fazendas verticais, empresas
que visam produzir alimentos de forma eficiente e automatizada em grandes
centros urbanos também podem se beneficiar do projeto.


%
% Elementos pós-textuais
%
\postextual

%
% Referências bibliográficas
%
\bibliography{refs}

\end{document}
