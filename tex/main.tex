\documentclass[
  12pt, % Tamaho da fonte
  openright, % Capítulos começam em página ímpar
  twoside, % Impressão em frente e verso
  a4paper, % Tamanho do papel
  english, % Idioma adicional para hifenização
  french, % Idioma adicional para hifenização
  spanish, % Idioma adicional para hifenização
  brazil % O último idioma é o principal do documento
]{abntex2}

%
% Pacotes
%
\usepackage{lmodern} % Usa a fonte Latin Modern
\usepackage[T1]{fontenc} % Seleção de códigos de fonte
\usepackage[utf8]{inputenc}	% Codificação do documento
\usepackage{indentfirst} % Indenta o primeiro parágrafo de cada seção
\usepackage{color} % Controle das cores
\usepackage{graphicx} % Inclusão de gráficos
\usepackage{microtype} % Para melhorias de justificação
\usepackage{pdfpages} % Para inclusão de arquivos PDF

%
% Pacotes de citações
%
\usepackage[brazilian,hyperpageref]{backref} % Páginas com as citações
\usepackage[alf]{abntex2cite}	% Citações no padrão ABNT

%
% Configurações de pacotes
%
\renewcommand{\backrefpagesname}{Citado na(s) página(s):~}
\renewcommand{\backref}{}
\renewcommand*{\backrefalt}[4]{
	\ifcase #1
		Nenhuma citação no texto.
	\or
		Citado na página #2.
	\else
		Citado #1 vezes nas páginas #2.%
	\fi
}

%
% Informações de dados para a capa e folha de rosto
%
\titulo{Projeto Integrador de Engenharia 1}
\autor{Grupo 01 (T01)}
\local{Brasília, DF}
\data{2024, v0.2.0}
\instituicao{Universidade de Brasília \par Faculdade do Gama}
\tipotrabalho{Relatório técnico}
% O preâmbulo deve conter o tipo do trabalho, o objetivo e o nome da instituição
\preambulo{
  Trabalho submetido à disciplina de Projeto Integrador de Engenharia 1 da
  Universidade de Brasília, ministrada pelo professor Diogo Garcia.
}
\orientador{Diogo Garcia}

% Informações do PDF
\makeatletter
\hypersetup{
  pdftitle={\@title},
  pdfauthor={\@author},
  pdfsubject={\imprimirpreambulo},
  pdfcreator={LaTeX with abnTeX2},
  pdfkeywords={abnt}{latex}{abntex}{projeto integrador de engenharia 1},
  colorlinks=true, % false: links em caixas; true: links coloridos
  linkcolor=blue, % Cor dos links internos
  citecolor=blue, % Cor dos links de citação
  filecolor=magenta, % Cor dos links para arquivos
  urlcolor=blue,
  bookmarksdepth=4
}
\makeatother

%
% Configurações de aparência do PDF final
%
\definecolor{blue}{RGB}{41,5,195} % Cor azul para os links

% Posiciona figuras e tabelas no topo da página quando adicionadas sozinhas
% em um página em branco. Ver https://github.com/abntex/abntex2/issues/170
\makeatletter
\setlength{\@fptop}{5pt}
\makeatother

% Possibilita criação de Quadros e Lista de quadros.
% Ver https://github.com/abntex/abntex2/issues/176
\newcommand{\quadroname}{Quadro}
\newcommand{\listofquadrosname}{Lista de quadros}

\newfloat[chapter]{quadro}{loq}{\quadroname}
\newlistof{listofquadros}{loq}{\listofquadrosname}
\newlistentry{quadro}{loq}{0}

% configurações para atender às regras da ABNT
\setfloatadjustment{quadro}{\centering}
\counterwithout{quadro}{chapter}
\renewcommand{\cftquadroname}{\quadroname\space}
\renewcommand*{\cftquadroaftersnum}{\hfill--\hfill}

% Ver https://github.com/abntex/abntex2/issues/176
\setfloatlocations{quadro}{hbtp}

%
% Espaçamentos entre linhas e parágrafos
%
\setlength{\parindent}{1.3cm}
\setlength{\parskip}{0.2cm}

%
% Início do documento
%
\begin{document}

\selectlanguage{brazil}
\frenchspacing

%
% Capa
%
\imprimircapa

%
% Folha de rosto
% (o * indica que haverá a ficha bibliográfica)
%
\imprimirfolhaderosto*

%
% Inserir folha de aprovação
%

% Isto é um exemplo de Folha de aprovação, elemento obrigatório da NBR
% 14724/2011 (seção 4.2.1.3). Você pode utilizar este modelo até a aprovação
% do trabalho. Após isso, substitua todo o conteúdo deste arquivo por uma
% imagem da página assinada pela banca com o comando abaixo:
%
% \begin{folhadeaprovacao}
%   \includepdf{folhadeaprovacao_final.pdf}
% \end{folhadeaprovacao}
%
\begin{folhadeaprovacao}
  \begin{center}
    {\ABNTEXchapterfont\large\imprimirautor}

    \vspace*{\fill}\vspace*{\fill}
    \begin{center}
      \ABNTEXchapterfont\bfseries\Large\imprimirtitulo
    \end{center}
    \vspace*{\fill}

    \hspace{.45\textwidth}
    \begin{minipage}{.5\textwidth}
        \imprimirpreambulo
    \end{minipage}%
    \vspace*{\fill}
   \end{center}

   Trabalho aprovado. \imprimirlocal, 24 de abril de 2024:

   \assinatura{\textbf{\imprimirorientador} \\ Orientador}
   \assinatura{\textbf{Rafael Rodrigues} \\ Coorientador}
   \assinatura{\textbf{Jungpyo Lee} \\ Coorientador}
   \assinatura{\textbf{Juliana Petrocchi} \\ Coorientador}
   \assinatura{\textbf{Ricardo Ajax} \\ Coorientador}

   \begin{center}
    \vspace*{0.5cm}
    {\large\imprimirlocal}
    \par
    {\large\imprimirdata}
    \vspace*{1cm}
  \end{center}
\end{folhadeaprovacao}

%
% Dedicatória
%
\begin{dedicatoria}
  \vspace*{\fill}
  \centering
  \noindent
  \textit{
    Dedicamos este trabalho aos nossos orientadores, que durante o semestre nos
    proporcionaram a melhor experiência para escrever um trabalho de qualidade.
  } \vspace*{\fill}
\end{dedicatoria}

%
% Resumo
%

% Ajusta o espaçamento dos parágrafos do resumo
\setlength{\absparsep}{18pt}
\begin{resumo}
  O seguidor de linha é uma categoria de robôs autônomos geralmente semelhantes
  a carros de corrida cujo objetivo é realizar um percurso no menor tempo
  possível. O carrinho seguidor de linha descrito nesse projeto vai um pouco
  além, além de ser capaz de realizar três percursos aleatórios no menor tempo
  possível, o carrinho precisa levar em si um ovo, que deve permanecer intacto
  ao final do percurso. Para isso, o projeto irá utilizar uma combinação de
  sensores, motores, algoritmos computacionais (incluindo inteligência
  artificial) e um hardwares de forma que o robô não saia da linha a ser
  disposta no chão e possa também calcular sua velocidade, distância percorrida
  e tempo de conclusão do percurso.

  \textbf{Palavras-chave}: Carrinho seguidor de linha. Engenharia. Projeto
    Integrador de Engenharia 1.
\end{resumo}

%
% Inserir lista de ilustrações
%
\pdfbookmark[0]{\listfigurename}{lof}
\listoffigures*
\cleardoublepage

%
% Inserir lista de quadros
%
\pdfbookmark[0]{\listofquadrosname}{loq}
\listofquadros*
\cleardoublepage

\end{document}
