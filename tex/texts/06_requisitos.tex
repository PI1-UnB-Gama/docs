\section{Requisitos}

Os requisitos de cada projeto variam muito dependendo de sua funcionalidade,
como exemplo, podemos citar o setor de armazenamento, no qual o produto requer
estrutura reforçada para grandes volumes e cargas, com dispositivos de segurança
integrados, com durabilidade, além de fácil uso e manutenção. Ou o setor de
agricultura vertical, que necessita eficiência energética, precisão de
navegação, integração com sistemas de controle e automação para irrigação entre
outras funções como identificação e controle de pragas.

Para este projeto em específico, podemos citar como requisitos:

%
% De acordo com a ABNT, a enumeração a seguir é considerada uma enumeração curta
% (pois nenhum item ultrapassa mais de três linhas), logo, cada item deve
% começar com letra minúscula e terminar com ponto e vírgula, exceto o último
% item que deve terminar com ponto final.
%
\begin{enumerate}
  \item percorrer distâncias no menor tempo possível;
  \item
    funcionalidades de monitoramento através do software (mostrar a trajetória
    percorrida, velocidade instantânea, aceleração instantânea e tempo de
    percurso);
  \item
    Durabilidade e qualidade dos materiais (não desmontar antes de completar os
    percursos).
  \item Segurança do ovo durante o percurso.
  \item Facilidade de uso (autonomia)
  \item
    Percorrer três percursos sem precisar remover ou alterar partes do hardware
    ou software.
  \item Permanecer sobre a linha a todo momento.
\end{enumerate}
