% Ajusta o espaçamento dos parágrafos do resumo
\setlength{\absparsep}{18pt}
\begin{resumo}
  O seguidor de linha, também chamado de \textit{follow line} ou
  \textit{line follower robot}, é uma categoria de robôs autônomos geralmente
  semelhantes a carros de corrida cujo objetivo é realizar um percurso no menor
  tempo possível seguindo uma linha previamente desenhada no chão. Este trabalho
  apresenta o projeto de um seguidor de linha que utiliza circuitos eletrônicos,
  sensores e softwares para realizar três percursos aleatórios. Além de
  percorrer os trajetos, o robô também deve ser capaz de carregar um ovo sem
  quebrá-lo, e apresentar dados de telemetria em algum dispositivo externo. O
  projeto foi desenvolvido por estudantes de engenharia da Universidade de
  Brasília durante a disciplina de Projeto Integrador de Engenharia 1,
  ministrada pelo professor Diogo Garcia, com o objetivo de abranger conteúdos
  muiltidisciplinares e unir diversas engenharias em um só trabalho. Essa
  proposta procura levar os alunos a entenderem como funciona o processo de
  criação de um produto e como se mobilizarem frente a um trabalho que necessita
  de outras áreas de conhecimento que não a de cada um individualmente.

  \textbf{Palavras-chave}:
    Seguidor de linha. Carrinho seguidor de linha. \textit{Follow line}.
    \textit{Line follower robot}. Engenharia. Projeto Integrador de Engenharia
    1. Universidade de Brasília. Faculdade do Gama. Diogo Garcia. Engenharia.
\end{resumo}
