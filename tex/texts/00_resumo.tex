% Ajusta o espaçamento dos parágrafos do resumo
\setlength{\absparsep}{18pt}
\begin{resumo}
  O seguidor de linha, também chamado de \textit{follow line} ou
  \textit{line follower robot}, é uma categoria de robôs autônomos geralmente
  semelhantes a carros de corrida cujo objetivo é realizar um percurso no menor
  tempo possível seguindo uma linha preta no chão. Este trabalho apresenta o
  projeto de um seguidor de linha que utiliza circuitos eletrônicos, sensores e
  softwares para realizar três percursos aleatórios. Além de percorrer os
  trajetos, o robô também deve ser capaz de carregar um ovo sem quebrá-lo, e
  apresentar dados de telemetria em algum dispositivo externo. O projeto
  foi desenvolvido por estudantes de engenharia da Universidade de Brasília
  durante a disciplina de Projeto Integrador de Engenharia 1, ministrada pelo
  professor Diogo Garcia.

  \textbf{Palavras-chave}:
    Seguidor de linha. Carrinho seguidor de linha. \textit{Follow line}.
    \textit{Line follower robot}. Engenharia. Projeto Integrador de Engenharia
    1. Universidade de Brasília. Faculdade do Gama. Diogo Garcia.
\end{resumo}
