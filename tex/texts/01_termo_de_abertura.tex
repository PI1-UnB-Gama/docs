\chapter{Termo de Abertura do Projeto}

\section{Dados do Projeto}

\begin{center}
  \begin{tabular}{|l|l|}
    \hline
    \textbf{Nome do Projeto} & Carrinho seguidor de linha \\
    \hline
    \textbf{Data de Abertura} & 17/04/2024 \\
    \hline
    \textbf{Código} & 1-A \\
    \hline
    \textbf{Patrocinador} & Universidade de Brasília \\
    \hline
    \textbf{Gerente} & Sarah Loriato Nazareth Franco (190020008) \\
    \hline
  \end{tabular}
\end{center}

\section{Objetivos}

\begin{description}
  \item [\textit{Specific} (específico):]
    Desenvolver um carrinho que seja capaz de percorrer completamente três
    trilhas marcadas no chão, sem auxílio externo para a sua movimentação e
    início de trajeto, exceto para ser iniciado.
  \item [\textit{Measurable} (mensurável):]
    O sucesso do carrinho será medido através de sua habilidade de completar os
    trajetos no menor tempo possível, sem causar dano ao ovo transportado. O
    registro de dados como trajetória percorrida, velocidade instantânea,
    aceleração instantânea, tempo de percurso e consumo energético ajudará a
    mensurar a performance.
  \item [\textit{Agreed} (acordado):]
    O projeto será acordado com as partes interessadas dentro da universidade,
    incluindo professores e alunos das diferentes engenharias da FGA, garantindo
    uma abordagem multidisciplinar no desenvolvimento do carrinho.
  \item [\textit{Realistic} (realista):]
    O objetivo é baseado na realidade do que pode ser alcançado pelos
    estudantes, utilizando conhecimentos de todas as engenharias da FGA,
    respeitando as restrições impostas pelos professores e pela universidade.
  \item [\textit{Time Bound} (Limitado no tempo):]
    O projeto tem um prazo determinado para sua finalização, alinhado com o
    calendário acadêmico e as datas estipuladas para os pontos de controle e
    apresentação final.
\end{description}
