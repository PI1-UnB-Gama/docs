\section{Estrutura Analítica}

Para o desenvolvimento do projeto, diversos fatores devem ser levados em
consideração de forma a facilitar a integração entre as frentes e a orientação
do projeto como um todo. A coleta e a análise dos dados coletados deverão ser
feitas de maneira precisa e coerente, utilizando técnicas e ferramentas que
permitam persistir e agrupar os valores obtidos em variáveis de banco de dados,
de forma a atender requisitos previamente estabelecidos para distância e tempo.

As atividades realizadas pelas frentes serão integradas e, por isto, deverão ser
executadas com rigor e exigem uma comunicação constante e eficaz entre as
partes, visando garantir uma melhor desenvolvimento do produto. As frentes de
software e hardware deverão trabalhar juntas no que diz respeito a integração
dos circuitos e componentes eletrônicos com os códigos e programas responsáveis
pelos movimentos e cálculos necessários para o projeto. A frente de consumo
energético trabalhará em conjunto com ambas, visando entender quais os
componentes e configurações necessárias para que o funcionamento do carrinho
esteja otimizado e o mais adequado possível. A frente de estrutura trabalhará em
contato com a frente de hardware e consumo energético, visando compreender as
necessidades estruturais para que o carrinho comporte os componentes eletrônicos
e energéticos necessários para seu funcionamento e para que os requisitos
previamente estabelecidos sejam atendidos.

%
% Comando para pular todo o conteúdo a seguir para a próxima página, para
% evitar que a figura fique no final da próxima seção.
%
\newpage

\begin{figure}[htb]
  \caption{\label{fig:eap} Estrutura Analítica do Projeto}

  \begin{center}
    \includegraphics[scale=0.3865]{../diagrams/eap.pdf}
  \end{center}

  \legend{Fonte: Elaborado pelos autores.}
\end{figure}
