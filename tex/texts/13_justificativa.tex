\section{Justificativa}

Como justificativa para o desenvolvimento deste trabalho, há de se pensar em
como a robótica e a automação já representam uma das formas mais eficientes e
tecnológicas presentes em comércios e indústrias. Com isso em mente, realizar um
projeto onde criaremos um seguidor de linha e que consiga carregar algo, pode
nos dar uma dimensão melhor em como aplicar isso na indústria.

Transporte e distribuição/organização de produtos em grandes volumes os armazéns
podem se beneficiar de um seguidor de linha para otimizar as operações de
movimentação de mercadorias. Inovação significativa no setor agrícola, trazendo
uma série de benefícios e oportunidades ainda em meio urbano. Automatizar
tarefas agrícolas com precisão e qualidade, como plantio, irrigação e colheita,
monitoramento e controle pode resultar em uma redução significativa nos custos
operacionais, contribuindo ainda para a sustentabilidade ambiental.

O seguidor de linha em meios logísticos se apresenta como uma ótima solução,
pois além de dispensar a mão de obra de um organizador, prioriza a eficiência
por meio de algoritmos que estão sempre em desenvolvimento. Para que esse
produto chegue nesse nível de qualidade a integração entre diversas engenharias
desempenha um papel um fundamental para sua construção e otimização.
