\chapter{Introdução}

Um seguidor de linha é um robô autônomo que segue uma linha preta no chão. O
objetivo é percorrer um percurso no menor tempo possível, evitando obstáculos e
mantendo-se dentro dos limites da pista. De acordo com a INATEL, a definição de
um seguidor de linha é:

\begin{citacao}
  \lbrack...\rbrack 
  \space
  uma categoria de robôs autônomos que, na maioria das vezes, se assemelham a
  carros de corrida. Através da combinação de motores, sensores e inteligência
  artificial, o principal objetivo é percorrer um determinado percurso no menor
  tempo possível. Os métodos utilizados, como mapeamento e a diferenciação entre
  retas e curvas, para que a estrutura mecânica tenha o melhor aproveitamento,
  podem ser empregados em diversos métodos de controle.
  \cite{INATEL:Seguidor-de-Linha}
\end{citacao}

Essa definição destaca a importância da integração de diferentes conceitos e
tecnologias para o desenvolvimento de um seguidor de linha eficiente. A
combinação de motores, sensores e algoritmos de controle é essencial para
garantir que o robô possa navegar com precisão e rapidez, evitando colisões e
desvios da linha de referência. Além disso, a capacidade de diferenciar entre
retas e curvas e de realizar mapeamentos do ambiente são aspectos fundamentais
para o desempenho do robô em diferentes cenários.

Uma vez que os robôs têm, frequentemente, um propósito de competição, a diretriz
que o projeto deve seguir varia da competição ou propósito final. No Brasil, não
existem quaisquer proibições sobre pequenos robôs seguidores de linha, as
possíveis legislações que podem abranger esse robô variam de acordo com seu
propósito, como se ele irá ou não ser comercializado \cite{Lei:8078:1990}, se
ele utiliza ou não formas sem fio para sua partida, de acordo com as
conformidades da ANATEL \cite{ANATEL:Manual-de-Orientacoes}, e se possuindo uma
câmera, ela faz registro e armazenamento das gravações \cite{Lei:12651:2012}.
Por fim, é responsabilidade dos construtores do robô não infringir leis como a
Lei de Propriedade Intelectual \cite{Lei:9279:1996} e respeitar a privacidade e
os direitos pessoais.

A indústria de robótica tem experimentado um crescimento significativo nos
últimos anos, impulsionado pelo avanço das tecnologias de automação e pela
demanda crescente por soluções inteligentes em diversos setores. Dentro desse
cenário, os seguidores de linha se destacam como uma aplicação prática e
educativa da robótica, especialmente popular entre entusiastas de tecnologia e
em ambientes educacionais. No entanto, o mercado atual para carrinhos
seguidor de linha é caracterizado por um alto custo de produtos similares, que
muitas vezes limita a acessibilidade para uma base mais ampla de consumidores.
Além disso, o número reduzido de empresas concorrentes pode restringir a
variedade e inovação de produtos disponíveis.

% Pular para a próxima página. Segundo as normas da ABNT,
% parágrafos não devem ser divididos entre páginas.
\clearpage

% TODO: Colocar referência correta do artigo citado abaixo.
Segundo Makedon, Mykhailenko e Vazov:

\begin{citacao}
  \lbrack...\rbrack 
  \space
  o nível de automação na indústria automotiva é geralmente muito mais alto do
  que em todos os outros setores. Desde 2014, um número significativo de robôs
  industriais foi entregue às empresas da indústria automobilística sul-coreana.
  \lbrack...\rbrack\space De acordo com especialistas da IFR, projetos para a
  produção de baterias para carros híbridos e veículos elétricos podem ser a
  fonte de um aumento significativo na densidade de robotização.
  \cite[p.~6]{Makedon:Dominants-and-Features-Market-Robotics}

\end{citacao}

Essas tendências globais destacam a importância da automação em vários setores,
inclusive na indústria de seguidores de linha, que pode se beneficiar da
integração de novas tecnologias para aprimorar a eficiência e a acessibilidade.

Neste trabalho, propomos o desenvolvimento de um carrinho seguidor de linha
autônomo, que seja capaz de percorrer trajetos pré-definidos sem auxílio
externo, além de transportar um ovo sem quebrá-lo, e enviar dados de telemetria,
como o trajeto percorrido, a velocidade instantânea, a aceleração instantânea, o
consumo energético e o tempo de percurso, para um dispositivo externo ao robô.
