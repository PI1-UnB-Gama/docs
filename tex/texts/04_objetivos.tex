\section{Objetivos}

Os objetivos do projeto são desenvolver um seguidor de linha autônomo que seja
capaz de percorrer trajetos pré-definidos sem auxílio externo, além de
transportar um ovo sem quebrá-lo. É possível descrever os objetivos do projeto
utilizando o acrônimo \textit{SMART}
\cite{University-of-California:SMART-Goals}, que serão detalhados a seguir.

\subsection{\textit{Specific} (específico)}

Desenvolver um carrinho que seja capaz de percorrer completamente três trilhas
marcadas no chão, sem auxílio externo para a sua movimentação e início de
trajeto, exceto para ser iniciado.

\subsection{\textit{Measurable} (mensurável)}

O sucesso do carrinho será medido através de sua habilidade de completar os
trajetos no menor tempo possível, sem causar dano ao ovo transportado. O
registro de dados como trajetória percorrida, velocidade instantânea, aceleração
instantânea, tempo de percurso e consumo energético ajudará a mensurar a
performance.

\subsection{\textit{Agreed} (acordado)}

O projeto será acordado com as partes interessadas dentro da universidade,
incluindo professores e alunos das diferentes engenharias da FGA, garantindo uma
abordagem multidisciplinar no desenvolvimento do carrinho.

\subsection{\textit{Realistic} (realista)}

O objetivo é baseado na realidade do que pode ser alcançado pelos estudantes,
utilizando conhecimentos de todas as engenharias da FGA, respeitando as
restrições de não utilização de sistemas prontos do mercado e focando na
originalidade do projeto.

\subsection{\textit{Time-bound} (limitado ao tempo)}

O projeto tem um prazo determinado para sua finalização, alinhado com o
calendário acadêmico e as datas estipuladas para os pontos de controle e
apresentação final.
