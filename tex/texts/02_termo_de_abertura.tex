\chapter{Termo de Abertura do Projeto}

\section{Dados do Projeto}

\begin{center}
  \begin{tabular}{|l|l|}
    \hline
    \textbf{Nome do Projeto} & Carrinho seguidor de linha \\
    \hline
    \textbf{Data de Abertura} & 17/04/2024 \\
    \hline
    \textbf{Código} & 1-A \\
    \hline
    \textbf{Patrocinador} & Universidade de Brasília \\
    \hline
    \textbf{Gerente} & Sarah Loriato Nazareth Franco (190020008) \\
    \hline
  \end{tabular}
\end{center}

\section{Objetivos}

Os objetivos do projeto são desenvolver um carrinho seguidor de linha autônomo
que seja capaz de percorrer trajetos pré-definidos sem auxílio externo, além de
transportar um ovo sem quebrá-lo. É possível descrever os objetivos do projeto
utilizando o acrônimo \textit{SMART} \cite{SMART-Goals:2017}, que seriam os
seguintes:

\subsection{\textit{Specific} (específico)}

Desenvolver um carrinho que seja capaz de percorrer completamente três
trilhas marcadas no chão, sem auxílio externo para a sua movimentação e
início de trajeto, exceto para ser iniciado.
 
\subsection{\textit{Measurable} (mensurável)}

O sucesso do carrinho será medido através de sua habilidade de completar os
trajetos no menor tempo possível, sem causar dano ao ovo transportado. O
registro de dados como trajetória percorrida, velocidade instantânea,
aceleração instantânea, tempo de percurso e consumo energético ajudará a
mensurar a performance.

\subsection{\textit{Agreed} (acordado)}

O projeto será acordado com as partes interessadas dentro da universidade,
incluindo professores e alunos das diferentes engenharias da FGA, garantindo
uma abordagem multidisciplinar no desenvolvimento do carrinho.

\subsection{\textit{Realistic} (realista)}

O projeto será baseado na realidade do que pode ser alcançado pelos estudantes,
utilizando conhecimentos de todas as engenharias da FGA, respeitando as
restrições impostas pelos professores e pela universidade.

\subsection{\textit{Time-bound} (limitado ao tempo)}

O projeto terá um prazo determinado para sua finalização, alinhado com o
calendário acadêmico e as datas estipuladas para os pontos de controle e
apresentação final.

\section{Mercado-alvo}

O carrinho seguidor de linha é um projeto que visa atender a demanda de
estudantes de engenharia que desejam desenvolver habilidades práticas e
multidisciplinares. Apesar disso, o projeto também pode ser utilizado por
empresas de logística e transporte, particularmente empresas de comércio
atacadista e armazéns no DF e região. Empresas de fazendas verticais, empresas
que visam produzir alimentos de forma eficiente e automatizada em grandes
centros urbanos também podem se beneficiar do projeto.
